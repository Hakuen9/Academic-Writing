\chapter{Conducted Analysis}

\addcontentsline{toc}{section}{5.1 Technical analysis}
\section*{\textbf{5.1 Technical analysis}}
The technical analysis consist of an intelligence analysis of Tay. This report cannot thoroughly conduct a technical analysis on the implementation of the artificial intelligence chatbot, since the project is closed sourced. Meaning that the code and implementation is not available for the general public. Therefore, alternative sources had to be consulted. The intelligence analysis was conducted by students of the Manipal Institute of Technology in India. This research provides examples of the how the chatbot learned from the input it received.

The conducted intelligence analysis consists of analyzing its 3000 most recently tweeted tweets. The findings which will be elaborated on are: Taxonomy Classification, Frequency Analysis and Term Frequency-Inverse Document Frequency vectors.

Firstly, the chatbot's tweets were categorized into predefined categories. “The Tay-Tweets corpus is analysed as a whole and five different categories are inferred, with corresponding confidence measures for each classification.”\cite{IntelligenceAnalysis}. These results led to an overview which primarily consisted of everyday topics. Which provides evidence of the audience of Tay, regular Twitter users.

Secondly, a frequency analysis was conducted on the tweets. Its vocabulary mainly consisted of basic phrases and slang used on Twitter. Some examples are: “chatting”, “human”, “DM”, “keep”\cite{IntelligenceAnalysis}. This analysis provides evidence of implementation of Tay. It supports the fact that the responses Tay generated were based on the questions it was asked.

Lastly, term frequency-inverse document frequency vectors were determined in order to research whether there exists a similarity between its responses and the questions it was asked. “We convert both of these corpora into a Term Frequency-Inverse Document Frequency vectors and then compute the cosine similarity.”\cite{IntelligenceAnalysis}. A cosine similarity of 0.9640545176 out of 1.0 was obtained. “A score closer to 1 indicates that both the questions and answers were highly similar”\cite{IntelligenceAnalysis}. This experiment provides evidence its responses and the questions Tay was asked.

In conclusion, the conducted intelligence analysis supports the fact that Tay uses the input of the user in order to formulate a fitting response.

\addcontentsline{toc}{section}{5.2 Ethical analysis}
\section*{\textbf{5.2 Ethical analysis}}
A question that arose in the problem identification section is, did the developers of Tay know about the vulnerability prior to the release of Tay. The ACM Code of Ethics says that a developer should “give comprehensive and thorough evaluations of computer systems and their impacts, including analysis of possible risks” \cite{ACM}. This means that if they had not foreseen this risk, they would still be obligated to foresee that Tay could eventually behave in a way they did not anticipate.

It cannot be guaranteed for certainty that all the developers were aware of these possible consequences. However, it can be concluded that Microsoft Corporation as a company was aware of the possible risks. The statement that Peter Lee gave on behalf of the company states that they “implemented a lot of filtering and conducted extensive user studies with diverse user groups.”\cite{statementpeterlee}.

When examining the results a question arised whether filtering was implemented. Filtering could not have prevent this situation completely, but by, for example, blocking anti-Semitic language the consequences of this exploit could have been prevented to a  large extend.

To summarize, an argument could be made about on which ethical theory Microsoft Corporation based their acting. It is safe to say that Microsoft knew of the possible consequences and that apparently a decision was made to release Tay. This is a common example of utilitarianism because the potential risk of this vulnerability being exploited was outweighed by the ability to collect a lot of real life data to improve their own algorithms.
