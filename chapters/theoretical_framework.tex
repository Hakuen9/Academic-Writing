\chapter{Theoretical Framework}

\section{Weak Artificial Intelligence}
In order to properly understand the very basis of the Tay artificial intelligence chatbot (AI chatbot), it is necessary to differentiate two major concepts in the AI paradigm. Weak and strong AI. As the names suggest the difference lies in the complexity of the algorithm.

Strong AI are able to think independently, and make decisions that are not predefined by the creator. The concept of strong AI is therefore quite the opposite of that of weak AI. Strong AI can be categorized as “systems that exhibit autonomous intelligence and decision making”\cite{AIdefinitions}.

Weak AI are built with the idea of replication/ duplication in mind. They are used for automatable tasks or tasks that can be repeated with ease. It is meant to perform a specific task without any independent thinking. Therefore, a weak AI can excel in a task and master it within a shorter time span than any human could. From this point of view, a weak AI could be viewed as an “intelligent” algorithm. However this is not true, since it can only perform a task well and it is not able to think independently and it is not able to “learn” a skill by itself.

There is no general consensus on the exact definition of weak AI. However, a definition that most researchers acquiesce upon is: “Weak AI is the concept that whatever the program is meant to do, it is merely trying to replicate or duplicate that function, and for most tasks that is sufficient”\cite{searleAIdefinition}. Since a rough definition of weak AI is sufficient in order to understand the working of the chatbot, the definition of Searle will be used in this report.

\newpage

\section{Ethical software development}
Over the years, multiple guidelines have been drawn up for Code Of Ethics by industry-trusted institutions and organizations. A subselection of this wide array of documents it listed below:
\begin{itemize}
	\item IEEE (Institute of Electrical and Electronics Engineers) Code of Ethics\cite{IEEE}
	\item ACM (Association for Computing Machinery) Code of Ethics and Professional Conduct\cite{ACM}
	\item CEI (Computer Ethics Institute) Ten Commandments of Computer Ethics\cite{CEI}
\end{itemize}
Although these are professional guidelines that are respected by the industry, they do not provide a succinct definition of the term "Computer Ethics". A succinct but well-defined definition of "Computer Ethics" is given by a Professor of Philosophy at DartMouth College: “Computer ethics is the analysis of the nature and social impact of computer technology and the corresponding formulation and justification of policies for the ethical use of such technology.”\cite{Moor}.
The definition of Moore will be used in this report, mainly because it speaks of the "social impact of computer technology". This fits in well with the problem statement given that Tay was released on a social media platform (i.e. Twitter).
