\chapter*{Appendix A}

\textbf{Peer Review - Voetman R. and Mellema J. - Academic Writing}

\underline{Jordi peer reviewed the following elements of the report:}

\begin{itemize}
    \item 3. Problem Identification
    \item 4.2 Theoretical Framework - Ethical software development
    \item 5.2 Conducted Analysis - Ethical Analysis
\end{itemize}

These chapters were written by Roy. I have summarized my findings below:

\underline{3. Problem Identification:}

‘e.g. a disability ability to post new tweets’ not entirely sure what the meaning of this sentence is.

“It was not Microsoft\textbf{’s} intention”. I believe it is stated in the academic writing hand-out, that contractions should be avoided when writing an academic report. Therefore I would suggest changing it to: “It was not Microsoft their intention”.

“if the conducted user studies really \textbf{where} extensive the vulnerability could possibly have been discovered at an earlier stage.” The word ‘where’ should be replaced with “were”, furthermore I would advise adding a comma between “extensive” and “the vulnerability”. Finally, I would remove “really” since it does not add any more meaning to the sentence.

“Did the developers \textbf{knew} about the vulnerability prior to the release of Tay?”
‘Did … knew’ should be ‘Did … know’.

How could Tay’s algorithms have been abused after the extensive testing of Microsoft \textbf{corp.}? I would change corp. to Corporation. As stated on the website of the University of New England: “Avoid using common abbreviations. It is best to write the full term in the text of your writing.”. Source: https://aso-resources.une.edu.au/academic-writing/usage/shortened-form-of-words/

\underline{4.2 Theoretical Framework: Ethical software development:}

I did not find any mistakes nor any erroneous grammatical structures in this subsection. I think this is a well-written subsection, in my opinion. Several guidelines regarding the development of Computer Ethics are listed. Which is more general.
Later on, Roy elaborates on the reason why the definition of professor J. H. Moor is sufficient enough. Which is in this is a specific example.The structure from general to specific fits the purpose of this part of the theoretical framework in my opinion. In conclusion, I think this is a well-written subsection.

\underline{5.2 Conducted Analysis: Ethical Analysis:}

“However, it can be concluded that Microsoft \textbf{inc.}” and “ethical theory Microsoft \textbf{inc.}” As stated earlier: “Avoid using common abbreviations. It is best to write the full term in the text of your writing.” This rule of thumb also applies here. However, I think it would be better to use Microsoft Corporation since using two different terms could be confusing. Furthermore it is also inconsistent.

“This \textbf{mean} that if they had not foreseen this \textbf{endanger}, they would still be obligated to foresee that Tay, because she learns from her own environment, could eventually behaved in a way they did not anticipate.”

Firstly, I think this sentence is quite lengthy. I think you could divide it into several sentences. Secondly, ‘mean’ should be ‘means’. Thirdly, I do not think ‘endanger’ is appropriate here, I would replace it with ‘endangerment’.

Furthermore, “could eventually behaved in a way”, seems to be incorrect. I think it should be: “could eventually behave in a way...”.

“However, it can be concluded that Microsoft inc. as a company \textbf{was} aware of the possible risks \textbf{was}.” Was is used twice here. The last one should be removed.

“When examining the results it can be concluded that filtering was either not implemented or to a very low extent.” I believe this is a logical fallacy. A hasty generalization in this case. The source code of Tay is closed sourced, meaning that the implementation of Tay is not available for the general public.

Making the statement that filtering was not implemented or to a \textbf{very} low extent. Is a hasty generalization. Because the implementation of Tay is not available to us. Therefore, a test framework could have been developed that excluded anti-semitic remarks. And included some other slurs or racial remarks. You can’t make such a statement because you know too little of the actual implementation, in my opinion.

I would also remove very, as it does not add any meaning (useful) to the sentence.

\newpage

\underline{Roy peer reviewed the following elements of the report:}

\begin{itemize}
    \item 1. Introduction
    \item 2. Problem Characterization
    \item 4.1 Theoretical Framework - Weak artificial intelligence
    \item 5.1 Conducted Analysis - Technical analysis
\end{itemize}

These chapters were written by Jordi. I have summarized my findings below:

\underline{1. Introduction:}

“another chatbot developed by Microsoft which is based in China.” when reading this sentence it seems like it Microsoft is based in China.

“Tay was taken offline since it started tweeting anti-semitic and racist remarks.” Before this sentence it is never mentioned that Tay is a twitter bot.

“Microsoft Corporation. requested for a thorough root cause analysis on Tay AI” Corporation does not need a dot at the end since it is not an abbreviation.
“requested for a” should be “requested a”.

“Finally, the recommendation will explain which elements of an artificial intelligence chatbot should be taken into careful consideration when developing and testing an intelligent chatbot.” Conclusion and recommendation are now merged and not two separate sections.

\underline{2. Problem Characterization:}

“This increase provides evidence of the risk of trolls, who could try to manipulate the chatbot.” Trolls is a very informal word, instead you could describe what a trolling user is.

“Moreover, It was clear that “ It without a capital I.

“Microsoft stated that trolls caused Tay to tweet those \textbf{thoughts}.” Is thought the right word to be used here? To prevent an argument that states if Tay can have thoughts or not it can be changed to the verb “messages”
Secondly, there is no citation to this statement.

\underline{4.1 Theoretical Framework - Weak artificial intelligence}

“Strong AI are able to think independently ...” Strong AI is singular but “are” is used which suggests using the plural form.

“The concept of strong AI is therefore quite the opposite of that of weak AI.”  Weak AI has not yet been defined, so this comparison can be interpreted as unclear

“Weak AI are built with ...“ Weak AI is singular but “are” is used which suggests using the plural form.

\underline{5.1 Conducted Analysis - Technical analysis}

“The conducted intelligence analysis consists of analyzing \textbf{the its 3000} most recently tweeted tweets.” The bold part is grammatically incorrect, I would remove the “its”.

“Firstly, \textbf{its} tweets were categorized into predefined categories.” This is the first sentence and “its” should refer to something in a past sentence, since there isn’t any this sentence is grammatically incorrect.

“Which provides evidence of audience \textbf{the Tay}. Everyday Twitter users.” Not really sure what the sentence should say but “the Tay” is not correct.

“Secondly, a \textbf{Frequency Analysis} ...” no need to use capitals in the bolded part.
